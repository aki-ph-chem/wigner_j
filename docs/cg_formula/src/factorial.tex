\subsection{階乗による表現}

Clebsch-Grodan係数$\bra{j_1m_1,j_2m_2}\ket{j_3m_3}$
は以下のとても長い式で機械的に計算することができる。

\begin{align}\label{cg_fact}
    &\bra{j_1m_1, j_2m_2}\ket{j_3m_3} \\ \notag 
    = 
    % delta
    &\delta_{m_1 + m_2, m_3}
    \sqrt{2j_3 + 1}
    \sqrt{\frac{(j_1 + j_2 - j_3)!(j_1 - j_2 + j_3)!(-j_1 + j_2 + j_3)!}{(j_1 + j_2 + j_3 + 1)!}}  \\ \notag
    % l_2
    &\times 
    \sqrt{(j_1 + m_1)!(j_1 -m_1)!(j_2 + m_2)!(j_2 - m_2)!(j_3 + m_3)!(j_3 - m_3)!} \\ \notag
    % l_3
    &\times
    \sum_z (-1)^z \frac{1}{z!(j_1 + j_2 - j_3 - z)!(j_1 - m_1 - z)!(j_2 + m_2 - z)!(j_3 - j_2 + m_1 + z)!(j_3 - j_1 - m_2 + z)!}
\end{align}

ここで$z$は$z$についての階乗の値が正となる範囲で取るものとする。

最初の$j_1,j_2,j_3$のみからなる部分を$\Delta_{j_1j_2j_3}$として

\begin{align}
    \Delta_{j_1j_2j_3} = \sqrt{2j_3 + 1}
    \sqrt{\frac{(j_1 + j_2 - j_3)!(j_1 - j_2 + j_3)!(-j_1 + j_2 + j_3)!}{(j_1 + j_2 + j_3 + 1)!}}  \\ \notag
\end{align}

のように取り出しtriangle factorと言うことがある。

ただし、この式には多くの階乗の計算が含まれているため、注意しないとすぐにオーバーフローしていしまう。
