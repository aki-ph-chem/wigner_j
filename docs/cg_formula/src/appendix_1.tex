\subsection{2つの表現の等価性}

方針: 式中の二項係数を階乗に直す



まず、シグマの前のルートの中身を計算する

\begin{align*}
    % L1
    &\frac{(2J_3 + 1)^2}{(2J_1 + 1)(2J_2 + 1)}
    \frac{F_{J_1 + J_2 - J_3}(J_1 + J_2 + J_3 + 1)}{F_{J_1 - J_2 + J_3}(J_1 + J_2 + J_3 + 1)}\\ 
    &\times \frac{1}{ F_{J_2 - J_1 + J_3}(J_1 + J_2 + J_3 + 1)} \\
    \\
    &\times \frac{F_{J_3 + m_3}(2J_3)}{F_{J_1 + m_1}(2J_1) F_{J_2 + m_2}(2J_2)} \\
    \\
    % 2
    &= \frac{(2J_3 + 1)^2}{(2J_1 + 1)(2J_2 + 1)} \frac{(J_1 + J_2 + J_3 + 1)!}{(J_1 + J_2 - J_3)!(2J_3 + 1)!} 
    \frac{(J_1 - J_2 + J_3)!(2J_2 + 1)!}{(J_1 + J_2 + J_3 + 1)!} \\
    \\
    &\times \frac{(J_2 - J_1 + J_3 )!(2J_1 + 1)!}{(J_1 + J_2 + J_3 + 1)!} \\
    \\
    &\times \frac{(2J_3)!}{(J_3 + m_3)!(J_3 - m_3)!} \frac{(J_1 + m_1)!(J_1 - m_1)!}{(2J_1)!} \frac{(J_2 + m_2)!(J_2 - m_2)!}{(2J_2)!} \\
    %
    \\
    % L3
    &= \frac{(2J_3 + 1)^2}{(2J_1 + 1)(2J_2 + 1)} \frac{(J_1 - J_2 + J_3)!(2J_2 + 1)!}{(J_1 + J_2 - J_3)!(2J_3 + 1)!} 
    \frac{(J_2 - J_1 + J_3)!(2J_2 + 1)!}{(J_1 + J_2 + J_3 + 1)!} \\
    \\ 
    &\times \frac{(2J_3)!}{(J_3 + m_3)!(J_3 - m_3)!} \frac{(J_1 + m_1)!(J_1 - m_1)!}{(2J_1)!} \frac{(J_2 + m_2)!(J_2 - m_2)!}{(2J_2)!} \\
    \\
    % L4
    &= (2J_3 + 1)\frac{(J_1 - J_2 + J_3)!}{(J_1 + J_2 - J_3)!} 
    \frac{(J_2 - J_1 + J_3)!}{(J_1 + J_2 + J_3 + 1)!} \\
    \\ 
    &\times \frac{(J_1 + m_1)!(J_1 - m_1)!(J_2 + m_2)!(J_2 - m_2)!}{(J_3 + m_3)!(J_3 - m_3)!}  \\
\end{align*}

よってシグマの前の項は以下のように計算できる

\begin{align}\label{out_sum}
    &\sqrt{2J_3 + 1} \sqrt{\frac{ (J_1 - J_2 + J_3)!(J_2 - J_1 + J_3)!}{(J_1 + J_2 - J_3)!(J_1 + J_2 + J_3 + 1)!} }\\ \notag
    &\times
    \frac{\sqrt{(J_1 - m_1)!(J_1 + m_1)!(J_2 + m_2)!(J_2 - m_2)!}}{\sqrt{(J_3 + m_3)!(J_3 - m_3)!}}
\end{align}

続いてシグマの中身を計算する

\begin{align*}
    &F_n(J_1 + J_2 - J_3)F_{J_2 + m_2 - n}(J_3 + m_3) F_{J_1 - m_1 - n}(J_3 - m_3) \\
    \\
    &= \frac{(J_1 + J_2 - J_3)!}{n!(J_1 + J_2 - J_3 - n)!} \frac{(J_3 + m_3)!}{(J_2 + m_2 - n)!(J_3 - J_2 + m_3 - m_2 + n)!} \\
    \\
    &\times \frac{(J_3 - m_3)!}{(J_1 - m_1 - n)!(J_3 - m_3 - J_1 + m_1 + n)!} \\
    \\
    &= (J_3 + m_3)! (J_3 - m_3)! (J_1 + J_2 - J_3)!\\
    \\
    &\times \frac{1}{n!(J_1 + J_2 - J_3 -n)!(J_2 + m_2 - n)!(J_3 - J_2 + m_1 + n)!} \\
    \\
    &\times \frac{1}{(J_1 - m_1 - n)!(J_3 - J_1 - m_2 + n)!} \\ 
    &(\because m_3 = m_1 + m_2)
\end{align*}

よってこれに$(-1)^n$を掛けて$n$について和をとり、先程計算した(\ref{out_sum})を掛けると

\begin{align*}
    &\sqrt{2J_3 + 1} \sqrt{\frac{ (J_1 - J_2 + J_3)!(J_2 - J_1 + J_3)!}{(J_1 + J_2 - J_3)!(J_1 + J_2 + J_3 + 1)!} }\\ 
    &\times
    \frac{\sqrt{(J_1 - m_1)!(J_1 + m_1)!(J_2 + m_2)!(J_2 - m_2)!}}{\sqrt{(J_3 + m_3)!(J_3 - m_3)!}} \\
    \\ 
    &\times (J_3 + m_3)! (J_3 - m_3)! (J_1 + J_2 - J_3)!\\
    \\
    &\times \sum_{n} \frac{(-1)^n}{n!(J_1 + J_2 - J_3 - n)!(J_2 + m_2 - n)!(J_3 - J_2 + m_1 + n)!(J_1 - m_1 - n)!(J_3 - J_1 - m_2 + n )!} \\
    \\
    %
    &=
    \sqrt{2J_3 + 1} \sqrt{\frac{(J_1 - J_2 + J_3)!(J_2 - J_1 + J_3)!(J_1 + J_2 - J_3)!}{(J_1 + J_2 + J_3 + 1)!} }\\ 
    \\
    &\times
    \sqrt{(J_1 - m_1)!(J_1 + m_1)!(J_2 + m_2)!(J_2 - m_2)!(J_3 + m_3)!(J_3 - m_3)!} \\
    \\ 
    &\times \sum_{n} \frac{(-1)^n}{n!(J_1 + J_2 - J_3 - n)!(J_2 + m_2 - n)!(J_3 - J_2 + m_1 + n)!(J_1 - m_1 - n)!(J_3 - J_1 - m_2 + n )!} \\
\end{align*}

と一致することが示される。
