\documentclass[dvipdfmx]{jsarticle}

% アンカーを作る
\usepackage[dvipdfmx]{hyperref}
\usepackage{pxjahyper}

\usepackage{comment}
\usepackage{amsmath}
\usepackage{amssymb}
\usepackage{bm}
\usepackage{physics}
\usepackage{comment}

\begin{document}
\title{Clebsch-Gordan係数の計算}
\author{Akira Shimizu}
\thispagestyle{empty}
\maketitle

% 目次
\tableofcontents
\clearpage

\section{はじめに}

Clebsch-Gordan係数は多くの場面で顔を出すため、値を知ることは重要である。
以下ではよく知られている階乗を用いて表現された公式とその公式と等価な二項係数による表現について説明する。

\section{Clebsch-Gordan係数の表現}
\subsection{階乗による表現}

Clebsch-Grodan係数$\bra{j_1m_1,j_2m_2}\ket{j_3m_3}$
は以下のとても長い式で機械的に計算することができる。

\begin{align}\label{cg_fact}
    &\bra{j_1m_1, j_2m_2}\ket{j_3m_3} \\ \notag 
    = 
    % delta
    &\delta_{m_1 + m_2, m_3}
    \sqrt{2j_3 + 1}
    \sqrt{\frac{(j_1 + j_2 - j_3)!(j_1 - j_2 + j_3)!(-j_1 + j_2 + j_3)!}{(j_1 + j_2 + j_3 + 1)!}}  \\ \notag
    % l_2
    &\times 
    \sqrt{(j_1 + m_1)!(j_1 -m_1)!(j_2 + m_2)!(j_2 - m_2)!(j_3 + m_3)!(j_3 - m_3)!} \\ \notag
    % l_3
    &\times
    \sum_z (-1)^z \frac{1}{z!(j_1 + j_2 - j_3 - z)!(j_1 - m_1 - z)!(j_2 + m_2 - z)!(j_3 - j_2 + m_1 + z)!(j_3 - j_1 - m_2 + z)!}
\end{align}

ここで$z$は$z$についての階乗の値が正となる範囲で取るものとする。

最初の$j_1,j_2,j_3$のみからなる部分を$\Delta_{j_1j_2j_3}$として

\begin{align}
    \Delta_{j_1j_2j_3} = \sqrt{2j_3 + 1}
    \sqrt{\frac{(j_1 + j_2 - j_3)!(j_1 - j_2 + j_3)!(-j_1 + j_2 + j_3)!}{(j_1 + j_2 + j_3 + 1)!}}  \\ \notag
\end{align}

のように取り出しtriangle factorと言うことがある。

ただし、この式には多くの階乗の計算が含まれているため、注意しないとすぐにオーバーフローしていしまう。

\subsection{二項係数による表現}

\begin{comment}
この式の階乗の分数の部分を二項係数(binomial)にした公式もある
多くのCG係数を計算するコードでは二項係数で計算しているものがほとんどであった。

漸化式を再帰的に計算するよりも級数で直接計算する方法の方が効率が良い
階乗(factorial)を計算するよりも二項係数(binomial)で計算する方が良い
\end{comment}


$\bra{J_1 J_2 m_1 m_2}\ket{J_3 m_3} \quad (m_1 + m_2 = m_3)$

は階乗ではなく二項係数を用いた表現では以下の級数で書ける

\begin{align}\label{cg_binomial}
    &\bra{J_1 J_2 m_1 m_2}\ket{J_3 m_3} =  \\ \notag
    %
    &\delta_{m_1+m_2, m_3}\sqrt{\frac{(2J_3 + 1)^2}{(2J_1 + 1)(2J_2 + 1)}} \\ \notag
    %
    &\times
    %
    \sqrt{\frac{ F_{J_1 + J_2 - J_3}(J_1 + J_2 + J_3 + 1)F_{J_3 + m_3}(2J_3)}
        {F_{J_1 - J_2 + J_3}(J_1 + J_2 + J_3 + 1)
    F_{J_2 - J_1 + J_3}(J_1 + J_2 + J_3 + 1)F_{J_1 + m_1}(2J_1)F_{J_2 + m_2}(2J_2) }} \\ \notag
    %
    &\times 
    %
     \sum_n
    (-1)^{n} F_n(J_1 + J_2 - J_3)F_{J_2 + m_2 - n}(J_3 + m_3)F_{J_1 - m_1 - n}(J_3 - m_3)
\end{align}

ここで、$F_{n}(m)$は二項係数で$_nC_k$よく知られている通り以下のように定義する。

\begin{align}
    F_m(n) = \frac{n!}{m!(n - m)!}
\end{align}

多くのClebsch-Gordan係数の実装では、(\ref{cg_fact})の階乗による表現ではなくてこの二項係数による実装を採用しているものが多かった。

\subsection{和を取る範囲について}

先程も述べたように$z$の範囲は$z$が絡んでくる階乗の中身(階乗される数値)の数値が負にならない範囲である。
ここで$z$が絡む階乗の中身すなわち(\ref{cg_fact})のシグマの中の数値が正となる条件から以下の連立不等式が得られる。

\begin{align}
    \begin{cases}
        z \ge 0  \tag{a}\\ 
        j_1 + j_2 - j_3 \ge z \\ 
        j_1 - m_1 \ge z \\
        j_2 + m_2 \ge z \\
        -j_3 + j_2 - m_1 \le z \\
        -j_3 + j_1 + m_2 \le z
    \end{cases}
\end{align}

%まず、$z \ge 0$より、最後の2式からは範囲は絞られない。
まず、$z$の最大値を考えよう。
$z \ge 0$の次から3個の条件式について考える。この3式の左辺のうち最小のものを考える。
$z$が、この最小の値から引かれても不等式を満たすならば、それよりも大きな値から$z$を引いた場合でも不等式を満たす。
これより、$z$の最大値は3式の左辺の最小値である。

\begin{align}
    z_{\text{max}} = \min\{j_1 + j_2 - j_3, j_1 - m_1, j_2 + m_2\}
\end{align}

最後に最小値を考えよう。
最後の二式の左辺が最大のものを考える。$z$が、この最大値から値を引かれても不等式を満たすならば、それよりも小さい値を$z$から引いても
不等式満たす。ここで第一式で$z \ge 0$を要求しているので結果として以下のように$0$と二式の左辺の値の最大値として得られる。

\begin{align}
    z_{\text{min}} = \max\{0, -j_3 + j_2 - m_1, -j_3 + j_1 + m_2\}
\end{align}


ただ、この最小値の議論をせずに、最小値を$0$として計算している実装も存在した。

しかし、最小値を$0$としたプログラムで計算を行うと間違った値が得られるため、最小値を$0$としている例が不適切で最小値の計算を行っているものが
正しいことが分かった。

\begin{comment}
また$n_{min},n_{max}$はそれぞれ以下で定義する
\begin{align}
    \begin{cases}
        n_{\text{min}} &= \max\{0,J_2 + m_2 - (J_3 + m_3), J_1 - m_1 - (J_3 - m_3)\} \\ 
                       &= \max\{0,J_2 - J_3 - m_1, J_1 - J_3 + m_2 \} \\ 
        n_{\text{max}} &= \min\{J_1 + J_2 - J_3, J_2 + m_2, J_1 - m_1\} \\ 
    \end{cases}
\end{align}
\end{comment}

\newpage
\subsection{値のテスト}

クレブシュゴルダン係数には規格直交性

\begin{align}\label{orthogonal_1}
    \sum_{m_1,m_2}\bra{JM}\ket{j_1 m_1 j_2 m_2}\bra{j_1 m_1 j_2 m_2}\ket{J^{\prime} M^{\prime}} 
    &= \bra{JM}\ket{J^{\prime} M^{\prime}} \\ \notag
    &= \delta_{J,J^{\prime}} \delta_{M,M^{\prime}}
\end{align}

が成立する。(\ref{orthogonal_1})の方の式に対して$J = J^{\prime}, M = M^{\prime}$とすると

\begin{align}\label{orhtogonal_2_2}
    \sum_{m_1,m_2} |\bra{JM}\ket{j_1 m_1 j_2 m_2}|^2 = 1
\end{align}

が成り立つ。


本来これは厳密に$1$であるが、浮動小数点演算によって計算を行うため、誤差が生じる。
そこで(\ref{orhtogonal_2_2})の右辺のと$1$との差の絶対値を$j_1,j_2$の関数として$\epsilon_{j_1,j_2}$として
以下で定義し、この値を評価する。

\begin{align}\label{error_cg}
    \epsilon_{j_1 j_2} = \left| 1 - \sum_{m_1,m_2}|\bra{J M}\ket{j_1 m_1 j_2 m_2}|^2 \right|
\end{align}

しかし、$\epsilon_{j_1 j_2}$はとても小さい数値で(論文では一番小さくて$10^{-33}$ほどのオーダー)であるので負の常用対数
すなわち$-\log \epsilon_{j_1 j_2}$の値を$\sqrt{j_1 j_2}$に対してplotして評価する。


\newpage
\section{Appendix}
\subsection{2つの表現の等価性}

方針: 式中の二項係数を階乗に直す



まず、シグマの前のルートの中身を計算する

\begin{align*}
    % L1
    &\frac{(2J_3 + 1)^2}{(2J_1 + 1)(2J_2 + 1)}
    \frac{F_{J_1 + J_2 - J_3}(J_1 + J_2 + J_3 + 1)}{F_{J_1 - J_2 + J_3}(J_1 + J_2 + J_3 + 1)}\\ 
    &\times \frac{1}{ F_{J_2 - J_1 + J_3}(J_1 + J_2 + J_3 + 1)} \\
    \\
    &\times \frac{F_{J_3 + m_3}(2J_3)}{F_{J_1 + m_1}(2J_1) F_{J_2 + m_2}(2J_2)} \\
    \\
    % 2
    &= \frac{(2J_3 + 1)^2}{(2J_1 + 1)(2J_2 + 1)} \frac{(J_1 + J_2 + J_3 + 1)!}{(J_1 + J_2 - J_3)!(2J_3 + 1)!} 
    \frac{(J_1 - J_2 + J_3)!(2J_2 + 1)!}{(J_1 + J_2 + J_3 + 1)!} \\
    \\
    &\times \frac{(J_2 - J_1 + J_3 )!(2J_1 + 1)!}{(J_1 + J_2 + J_3 + 1)!} \\
    \\
    &\times \frac{(2J_3)!}{(J_3 + m_3)!(J_3 - m_3)!} \frac{(J_1 + m_1)!(J_1 - m_1)!}{(2J_1)!} \frac{(J_2 + m_2)!(J_2 - m_2)!}{(2J_2)!} \\
    %
    \\
    % L3
    &= \frac{(2J_3 + 1)^2}{(2J_1 + 1)(2J_2 + 1)} \frac{(J_1 - J_2 + J_3)!(2J_2 + 1)!}{(J_1 + J_2 - J_3)!(2J_3 + 1)!} 
    \frac{(J_2 - J_1 + J_3)!(2J_2 + 1)!}{(J_1 + J_2 + J_3 + 1)!} \\
    \\ 
    &\times \frac{(2J_3)!}{(J_3 + m_3)!(J_3 - m_3)!} \frac{(J_1 + m_1)!(J_1 - m_1)!}{(2J_1)!} \frac{(J_2 + m_2)!(J_2 - m_2)!}{(2J_2)!} \\
    \\
    % L4
    &= (2J_3 + 1)\frac{(J_1 - J_2 + J_3)!}{(J_1 + J_2 - J_3)!} 
    \frac{(J_2 - J_1 + J_3)!}{(J_1 + J_2 + J_3 + 1)!} \\
    \\ 
    &\times \frac{(J_1 + m_1)!(J_1 - m_1)!(J_2 + m_2)!(J_2 - m_2)!}{(J_3 + m_3)!(J_3 - m_3)!}  \\
\end{align*}

よってシグマの前の項は以下のように計算できる

\begin{align}\label{out_sum}
    &\sqrt{2J_3 + 1} \sqrt{\frac{ (J_1 - J_2 + J_3)!(J_2 - J_1 + J_3)!}{(J_1 + J_2 - J_3)!(J_1 + J_2 + J_3 + 1)!} }\\ \notag
    &\times
    \frac{\sqrt{(J_1 - m_1)!(J_1 + m_1)!(J_2 + m_2)!(J_2 - m_2)!}}{\sqrt{(J_3 + m_3)!(J_3 - m_3)!}}
\end{align}

続いてシグマの中身を計算する

\begin{align*}
    &F_n(J_1 + J_2 - J_3)F_{J_2 + m_2 - n}(J_3 + m_3) F_{J_1 - m_1 - n}(J_3 - m_3) \\
    \\
    &= \frac{(J_1 + J_2 - J_3)!}{n!(J_1 + J_2 - J_3 - n)!} \frac{(J_3 + m_3)!}{(J_2 + m_2 - n)!(J_3 - J_2 + m_3 - m_2 + n)!} \\
    \\
    &\times \frac{(J_3 - m_3)!}{(J_1 - m_1 - n)!(J_3 - m_3 - J_1 + m_1 + n)!} \\
    \\
    &= (J_3 + m_3)! (J_3 - m_3)! (J_1 + J_2 - J_3)!\\
    \\
    &\times \frac{1}{n!(J_1 + J_2 - J_3 -n)!(J_2 + m_2 - n)!(J_3 - J_2 + m_1 + n)!} \\
    \\
    &\times \frac{1}{(J_1 - m_1 - n)!(J_3 - J_1 - m_2 + n)!} \\ 
    &(\because m_3 = m_1 + m_2)
\end{align*}

よってこれに$(-1)^n$を掛けて$n$について和をとり、先程計算した(\ref{out_sum})を掛けると

\begin{align*}
    &\sqrt{2J_3 + 1} \sqrt{\frac{ (J_1 - J_2 + J_3)!(J_2 - J_1 + J_3)!}{(J_1 + J_2 - J_3)!(J_1 + J_2 + J_3 + 1)!} }\\ 
    &\times
    \frac{\sqrt{(J_1 - m_1)!(J_1 + m_1)!(J_2 + m_2)!(J_2 - m_2)!}}{\sqrt{(J_3 + m_3)!(J_3 - m_3)!}} \\
    \\ 
    &\times (J_3 + m_3)! (J_3 - m_3)! (J_1 + J_2 - J_3)!\\
    \\
    &\times \sum_{n} \frac{(-1)^n}{n!(J_1 + J_2 - J_3 - n)!(J_2 + m_2 - n)!(J_3 - J_2 + m_1 + n)!(J_1 - m_1 - n)!(J_3 - J_1 - m_2 + n )!} \\
    \\
    %
    &=
    \sqrt{2J_3 + 1} \sqrt{\frac{(J_1 - J_2 + J_3)!(J_2 - J_1 + J_3)!(J_1 + J_2 - J_3)!}{(J_1 + J_2 + J_3 + 1)!} }\\ 
    \\
    &\times
    \sqrt{(J_1 - m_1)!(J_1 + m_1)!(J_2 + m_2)!(J_2 - m_2)!(J_3 + m_3)!(J_3 - m_3)!} \\
    \\ 
    &\times \sum_{n} \frac{(-1)^n}{n!(J_1 + J_2 - J_3 - n)!(J_2 + m_2 - n)!(J_3 - J_2 + m_1 + n)!(J_1 - m_1 - n)!(J_3 - J_1 - m_2 + n )!} \\
\end{align*}

と一致することが示される。


\end{document}
